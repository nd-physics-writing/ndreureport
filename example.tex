%%%%%%%%%%%%%%%%%%%%%%%%%%%%%%%%%%%%%%%%%%%%%%%%%%%%%%%%%%%%%%%%
%
% Example of use of ND REU Report class
%
% Authors: Patrick J. Fasano and Charlotte M. Wood
%
% License: Creative Commons Attribution 4.0 International
% (https://creativecommons.org/licenses/by/4.0/)
%
% SPDX-License-Identifier: CC-BY-4.0
%
%%%%%%%%%%%%%%%%%%%%%%%%%%%%%%%%%%%%%%%%%%%%%%%%%%%%%%%%%%%%%%%%

\documentclass{ndreureport}

% This package is only used to provide filler text for this example.
% Do not include it in your report document.
\usepackage{lipsum}

%%%%%%%%%%%%%%%%%%%%%%%%%%%%%%%%%%%%%%%%%%%%%%%%%%%%%%%%%%%%%%%%
% Title/Author/Advisor for Title Page
%%%%%%%%%%%%%%%%%%%%%%%%%%%%%%%%%%%%%%%%%%%%%%%%%%%%%%%%%%%%%%%%

\author{Heinz Doofenshmirtz}
\title{The Other-Dimension-Inator}
\advisors{Prof. Kevin Destructicon}


\begin{document}
\maketitle

%%%%%%%%%%%%%%%%%%%%%%%%%%%%%%%%%%%%%%%%%%%%%%%%%%%%%%%%%%%%%%%%
% Abstract
%%%%%%%%%%%%%%%%%%%%%%%%%%%%%%%%%%%%%%%%%%%%%%%%%%%%%%%%%%%%%%%%

\begin{abstract}
    \lipsum[1]
\end{abstract}

%%%%%%%%%%%%%%%%%%%%%%%%%%%%%%%%%%%%%%%%%%%%%%%%%%%%%%%%%%%%%%%%\
% Main Body
%%%%%%%%%%%%%%%%%%%%%%%%%%%%%%%%%%%%%%%%%%%%%%%%%%%%%%%%%%%%%%%%

\section{Introduction}

Here is an example of how you reference papers. The BibTeX citations go in your
.bib file. This is some information \cite{sample2021}.

\subsection{Subsection}

\lipsum[2]

Let's show an example of how you add in math. You can add it inline like this:
$a = b + c$, or you can add it outside of the text.

\begin{equation}
    a = b + c
    \label{eq:a}
\end{equation}

\subsection{Subsection}

\lipsum[3]

Now we'll do inline references to figures/tables/equations that you have in your
paper. If you look at Figure \ref{fig:test} you'll see a sample figure. This has
nothing to do with Equation \ref{eq:a}.

\section{Background}

Here's a sample table using the \texttt{tabular} environment:
\begin{center}
    \begin{tabular}{c|l|r}
        Col 1 & Col 2 & Col 3 \\ \hline
        A & a & 1.0 \\
        B & b & 2.0 \\
        C & c & 3.0
    \end{tabular}
\end{center}
It will be set directly inline, in the flow of your text. If you want the table
to ``float'' to the best place, wrap it in a \texttt{table} environment, like
Table~\ref{tab:test}.

\begin{table}[t]
    \centering
    \begin{tabular}{c|lr}
        Col $\alpha$ & Col $\beta$ & Col $\gamma$ \\ \hline
        X & a & 1.0 \\
        Y & b & 2.0 \\
        Z & c & 3.0
    \end{tabular}
    \caption{\label{tab:test}A floating table's caption goes here.}
\end{table}

\lipsum[5][1-3]

\section{Methods}

\lipsum[6]

\section{Results}

\begin{figure}
    \centering
    \includegraphics[width=0.9\textwidth]{figures/fig_1.eps}
    \caption{\label{fig:float}This is a floating figure.}
\end{figure}

A sample figure is shown in Fig.~\ref{fig:float}. In general, \LaTeX will figure
out the best place to put your figure as long as you put it at the beginning of
the section where you first reference it. However, you can override the
placement by providing extra options to control placement. For
Fig.~\ref{fig:test}, the \texttt{ht!} options tell \LaTeX where to put the
figure (here (\texttt{h}), top (\texttt{t}), and override what it thinks is best
(\texttt{!})).
\begin{figure}[ht!]
    \centering
    \includegraphics[width=0.9\textwidth]{figures/fig_1.eps}
    \caption{\label{fig:test}Figure caption goes here}
\end{figure}

\lipsum[6]

\section{Conclusion}

\lipsum[8]

%%%%%%%%%%%%%%%%%%%%%%%%%%%%%%%%%%%%%%%%%%%%%%%%%%%%%%%%%%%%%%%%
% Bibliography
%%%%%%%%%%%%%%%%%%%%%%%%%%%%%%%%%%%%%%%%%%%%%%%%%%%%%%%%%%%%%%%%
\bibliography{example}

\end{document}
